\begin{spacing}{1}
\begin{center}
		\large\textbf{\JdTesis}
	\end{center}
	\normalsize
	\begin{adjustwidth}{-0.2cm}{}
		\ifthenelse{\boolean{bMaster}}{
		
		\begin{tabular}{lcp{0.9\linewidth}}
		Nama Mahasiswa &:& \NamaMahasiswa\\
			NRP &:&\NrpMahasiswa\\
			Pembimbing &:& 1. \PbSatu\\
			\ifthenelse{\boolean{PembimbingDua}}{& & 2. \PbDua\\}{}
			
			\ifthenelse{\boolean{PembimbingTiga}}{& & 3. \PbTiga\\}{}
			
			
			
		\end{tabular}
	}{
	\begin{tabular}{lcp{0.7\linewidth}}
		Nama Mahasiswa &:& \NamaMahasiswa\\
		NRP &:&\NrpMahasiswa\\
		Promotor &:&  \PbSatu\\
		\ifthenelse{\boolean{PembimbingTiga}}{
		\ifthenelse{\boolean{PembimbingDua}}{Co. Promotor&: & 1. \PbDua\\}{}
		\ifthenelse{\boolean{PembimbingTiga}}{& & 2. \PbTiga\\}{}
	}
{
	\ifthenelse{\boolean{PembimbingDua}}{\hspace{5ex}Co. Promotor&: & \PbDua\\}{}
	
}
	\end{tabular}

}

	
	\end{adjustwidth}
	\vspace{2ex}
	\begin{center}
		\Large\textbf{ABSTRAK}
	\end{center}
	\vspace{1ex}	
%Tulis Abstrak disini
Explainable AI atau lebih banyak dikenal dengan sebutan XAI adalah sebuah metode untuk mengatasi permasalahan kotak hitam pada model AI. Yaitu sebuah permasalahan dimana sebuah model AI tidak transparan dan tidak bisa dijelaskan korelasi antara inpu-output nya. Pada riset ini dilakukan serangkaian percobaan untuk mengimplementasikan metode XAI pada model forecasting dan deteksi anomali. Persebaran panas pada boiler PLTU dijadikan sebagai studi kasus. Secara khusus, model yang diaplikasikan menggunakan arsitektur encoder-decoder, digunakan untuk melakukan forecasting temperature dari pipa-pipa reheater pada boiler dan untuk mendeteksi terjadinya anomali. Metode XAI yang digunakan berbasis pada feature importance. Output dari penelitian ini adalah informasi mengenai variabel input yang paling signifikan mempengaruhi output dari model AI.

%Tulis Kata Kunci disini
\vspace{2ex}
\textbf{Kata kunci }: XAI, Forecasting, Encoder-Decoder, Anomaly Detection
	
\end{spacing}