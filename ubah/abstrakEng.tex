\begin{spacing}{1}
\begin{center}
		\large\textbf{\JdTesisEng}
	\end{center}
	\normalsize
	\begin{adjustwidth}{-0.2cm}{}
		\begin{tabular}{lcp{0.9\linewidth}}
		By &:& \NamaMahasiswa\\
			Student Identity Number &:&\NrpMahasiswa\\
			Supervisor &:& 1. \PbSatu\\
			\ifthenelse{\boolean{PembimbingDua}}{& & 2. \PbDua\\}{}
			\ifthenelse{\boolean{PembimbingTiga}}{& & 3. \PbTiga\\}{}
		\end{tabular}
	\end{adjustwidth}
	\vspace{2ex}
	
	\begin{center}
		\Large\textbf{ABSTRACT}
	\end{center}
	\vspace{1ex}	
%Tulis Abstrak disini
Icar is an autonomous vehicle developed by Institut Teknologi Sepuluh Nopember (ITS) to autonomously navigate the campus environment. Its navigation system relies on pose estimation using sensor fusion between odometry and the Global Navigation Satellite System (GNSS). However, GNSS signals are prone to degradation in environments with obstacles such as tall buildings and trees, while odometry suffers from inaccuracies caused by wheel slippage, encoder drift, and inertial sensor errors. These limitations result in unreliable localization, potentially causing Icar to deviate from its designated trajectory.

To address these issues, this research proposes an enhanced navigation system by integrating a stereo depth camera and LIDAR. Road detection is performed using a semantic segmentation model based on deep learning, enabling Icar to distinguish road surfaces from non-road areas. Pose refinement is carried out using graph-based optimization within a SLAM framework, incorporating data from the depth camera and LIDAR processed via the Iterative Closest Point (ICP) algorithm. The proposed system improves localization robustness and offers a GNSS-independent alternative for safe and accurate autonomous navigation. The final system is expected to reduce dependency on GNSS and increase Icar's reliability in semi-structured or dynamic environments.

%Tulis Kata Kunci disini
\vspace{2ex}
\textbf{Keyword}: Autonomous vehicle, Icar, GNSS, Odometry, Pose Estimation, Stereo Depth Camera, Semantic Segmentation, SLAM, ICP, Graph-Based Optimization, Deep Learning
\end{spacing}